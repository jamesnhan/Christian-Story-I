\documentclass{report}
\usepackage{hyperref}
\hypersetup{colorlinks=false, linkcolor=black, urlcolor=black, linktoc=all}
\usepackage{helvet}
\renewcommand{\familydefault}{\sfdefault}

\title{The Bible: An Introduction}
\author{Jerry L. Sumney}

\begin{document}

\maketitle

\tableofcontents{}

\part{What Is The Bible, And How Did It Come About?}

    \chapter{The Bible}
        \section{The Bible: A Collection}
            The Bible is a collection of books written by several different authors over many hundreds of years in three languages: Hebrew, Aramaic, and Greek in several different kinds of writings such as narratives, letters, psalms, poetry, and ``an apocalyptic.'' Some books even have multiple authors.
        \section{The Emergence of the Canon}
        The Bible was collected over several hundred years. The process of the collection included a great deal of thought and dicussion, which culminated in \textbf{39} books of the \textbf{Protestant Old Testament} (the \textbf{Hebrew Bible} counts the same writings as \textbf{24} books by combining parts of books into one) and the \textbf{27} books of the \textbf{New Testament}. \textbf{Canon} designates a collection of writings that carries authority in a given religious community and comes
        from the Greek word \textit{kanon}, which means ``measuring stick.'' Canon is a standard for the religious community to evaluate beliefs, practices, and ethical behavior. Boundaries also help guide their beliefs and practices by limiting membership. Groups without boundaries render membership meaningless. Standards and boundaries is required for a well-defined group.
        \section{The Canon of the Hebrew Bible}
        Both the Jewish community and the Christian community developed a canon. First the Hebrew Bible developed; then, the New Testament arose.

        The process of collecting the books of the Hebrew Bible was complicated due to the fact that many books showed evidence of having multiple authors and sources from others. This suggests that the authors would take what they found in other books and interpret them with religious or theological notions to reveal the subjects' relationship with God. Another difficulty stemmed from the fact that several books often were not completed until after the prophet's death, which is indicated through the use of third person.

        Due to these difficulties, the actual compilation did not begin until about the sixth century B.C.E. The story of King Josiah (640\texttt{-}609 B.C.E.) shows how late the process started. In 2 Kings 22:8\texttt{-}13, Josiah found ``The Book of the Law'' in a back room of the dilapidated temple of God in Jerusalem while refurbishing it. After the prophet Huldah verified it as the word of God, Josiah began a reform based on the text. This shows that there was no collective authoratative canon until at least after this time, and the assembling did not begin in earnest until at least during the \textbf{exile}.

        The first evidence was after the exile when Ezra gathered the people in Jerusalem after being released from the Persian captivity and read to the masses the ``book of the law of Moses'' (Nehemia 8:1\texttt{-}3). Although Nehemia describes the book as what God gave Moses, there's no evidence that it existed in written form until some time after the exile. While we don't know what was in the collection Ezra read, it probably included much of the \textbf{Pentateuch}, the first five books of the Bible. In the Hebrew Bible, this is called the \textbf{Torah}. Over the next \textbf{250} years, other books were collected and began to be revered. By the mid-second century B.C.E., the book called Ecclesiasticus or Sirach (part of the Apocrypha) could refer to a collection of writings divided into three groups: ``the Law, the Prophets, and the other books.''

        When the \textbf{Dead Sea Scrolls} were found in \textbf{1947}, we found copies of every book in the Hebrew Bible except Esther made just before and during the time of Jesus.
        \section{The Christian Canon}
        \section{Conclusion}

    \chapter{From Then to Now}
    \chapter{Inspiration}

\part{What Is The Story Of The Hebrew Bible?}

    \chapter{The Pentateuch, Part I}
    \chapter{The Pentateuch, Part II}
    \chapter{The Israelites Tell Their Story}
    \chapter{``Thus Says The Lord''}
    \chapter{An Alternative Worldview}
    \chapter{Israel's Response to God}
    \chapter{Between the Testaments}

\part{What Is The Story Of The New Testament?}

    \chapter{The Gospels}
    \chapter{Four Views of One Jesus}
    \chapter{The Story Continues}
    \chapter{The Pauline Letters}
    \chapter{The Disputed Pauline Letters}
    \chapter{Hebrews and the General Epistles}
    \chapter{Revelation}

\part{Epilogue}

    \chapter{The Bible Today}

\end{document}
