\documentclass{report}
\usepackage{hyperref}
\hypersetup{colorlinks=false, linkcolor=black, urlcolor=black, linktoc=all}
\usepackage[british]{babel}
\usepackage[nodayofweek]{datetime}
\usepackage{helvet}
\renewcommand{\familydefault}{\sfdefault}
\usepackage{longtable}

\title{The Bible: An Introduction}
\author{Jerry L. Sumney}
\date{\longdate{\today}}

\begin{document}

\maketitle

\tableofcontents{}

\part{What Is The Bible, And How Did It Come About?}

    \chapter{The Bible}
        \section{The Bible: A Collection}
            The Bible is a collection of books written by several different authors over many hundreds of years in three languages: Hebrew, Aramaic, and Greek in several different kinds of writings such as narratives, letters, psalms, poetry, and ``an apocalyptic.'' Some books even have multiple authors.
            
        \section{The Emergence of the Canon}
        The Bible was collected over several hundred years. The process of the collection included a great deal of thought and dicussion, which culminated in \textbf{39} books of the \textbf{Protestant Old Testament} (the \textbf{Hebrew Bible} counts the same writings as \textbf{24} books by combining parts of books into one) and the \textbf{27} books of the \textbf{New Testament}. \textbf{Canon} designates a collection of writings that carries authority in a given religious community and comes
        from the Greek word \textit{kanon}, which means ``measuring stick.'' Canon is a standard for the religious community to evaluate beliefs, practices, and ethical behavior. Boundaries also help guide their beliefs and practices by limiting membership. Groups without boundaries render membership meaningless. Standards and boundaries is required for a well-defined group.

        \section{The Canon of the Hebrew Bible}
        Both the Jewish community and the Christian community developed a canon. First the Hebrew Bible developed; then, the New Testament arose.

        The process of collecting the books of the Hebrew Bible was complicated due to the fact that many books showed evidence of having multiple authors and sources from others. This suggests that the authors would take what they found in other books and interpret them with religious or theological notions to reveal the subjects' relationship with God. Another difficulty stemmed from the fact that several books often were not completed until after the prophet's death, which is indicated through the use of third person.

        Due to these difficulties, the actual compilation did not begin until about the sixth century B.C.E. The story of King Josiah (640\texttt{-}609 B.C.E.) shows how late the process started. In 2 Kings 22:8\texttt{-}13, Josiah found ``The Book of the Law'' in a back room of the dilapidated temple of God in Jerusalem while refurbishing it. After the prophet Huldah verified it as the word of God, Josiah began a reform based on the text. This shows that there was no collective authoratative canon until at least after this time, and the assembling did not begin in earnest until at least during the \textbf{exile}.

        The first evidence was after the exile when Ezra gathered the people in Jerusalem after being released from the Persian captivity and read to the masses the ``book of the law of Moses'' (Nehemia 8:1\texttt{-}3). Although Nehemia describes the book as what God gave Moses, there's no evidence that it existed in written form until some time after the exile. While we don't know what was in the collection Ezra read, it probably included much of the \textbf{Pentateuch}, the first five books of the Bible. In the Hebrew Bible, this is called the \textbf{Torah}. Over the next \textbf{250} years, other books were collected and began to be revered. By the mid-second century B.C.E., the book called Ecclesiasticus or Sirach (part of the Apocrypha) could refer to a collection of writings divided into three groups: ``the Law, the Prophets, and the other books.''

        When the \textbf{Dead Sea Scrolls} were found in \textbf{1947}, we found copies of every book in the Hebrew Bible except Esther made just before and during the time of Jesus.

        The other important evidence for what Jews were reading as canon is the \textbf{Septuagint}, which is a translation of the Hebrew Bible into Greek near the end of the second century B.C.E. and included all of the books of the Hebrew Bible as well as the Apocrypha. While it was not completely accepted, it was known enough that it could be referenced and understood by other Jews. The canon of the Hebrew Bible was the result of \textbf{10} centuries of work and thought.

        \section{The Christian Canon}
        The Protestant Old Testament is the same as the Hebrew Bible except for order and numbering. The Roman Catholic Bible includes as \textbf{deuterocanonical} the books called the Apocrypha, which the Jews did not include as such because the books were not originally written in Hebrew. These books plus additions to Daniel and Esther were part of the Septuagint, and they remained within the canon in some parts of Christian tradition because of this fact. It also caused them to be included in the \textbf{Vulgate}, the fifth-century translation of the Bible into Latin. The Greek Orthodox Bible includes two additional books, 3 Maccabees and 2 Esdras, books also preserved in Greek and known widely in the early church.

        The earliest book of the New Testament was not written until around 50 C.E. The latest of the New Testament books was composed around 125 C.E. Many of the books were written anonymously. These books' authority was questioned due to the fact that these books were not directly written by the \textbf{apostles}. The apostles were said to be the ones who most clearly understood the meaning behind Jesus' words and actions, and as such, were viewed as the authorities on these topics.

        The first Christian canon was circulated by Marcion. He spoke about how he believed the God of the Hebrew Bible was not the Father of Jesus Christ. The god of Israel was too violent and vindictive to be the kind, forgiving, loving God of Jesus Christ. This canon was vehemently rejected by the larger body of the church.o

        Thus, the apostle's authority was spawned. A text must have some connection to an apostle through direct authorship or through some other relationship. Additionally, a piece must be known and widely used across the Christian world and agree with the ``rule of faith'' (it had to cohere with the beliefs of the early church).

        There was always dispute about the books, but by 200, nearly everyone accepted 10 letters of Paul, 4 Gospels, Acts, 1 Peter, and 1 John. By the fourth century, nearly everyone acepts 22 books: the 4 Gospels, Acts, 13 letters of Paul, Hebrews, 1 John, 1 Peter, and Revelation. For several centuries, the \textbf{Vulgate}, which included the 27 current books of the New Testament became the Bible of the church.

        Finally, in the sixteenth century, the official declaration of the church was released when Martin Luther questioned the value and teachings of the four New Testament books: Hebrews, James, Jude, and Revelation. The Counter-Reformation \textbf{Council of Trent} declared it an article of faith that one accept the current 27 books as canonical. Most accepted this, but the Catholic Church retained the Apocrypha, while the Protestants did not include the Apocrypha.

        \section{Key Terms}
        \begin{center}
            \begin{longtable}{| p{5cm} | p{10cm} |}
            \hline
            \textbf{Apocrypha} & The seven books included as deuterocanonical in the Roman Catholic Bible, which are Tobit, Judith, 1\texttt{-}2 Maccabees, Wisdom, Sirach, and Baruch.\\ \hline
            \textbf{Apostles} & The central authorities within the church after the death of Jesus. They were those who were closest to Jesus during his life.\\ \hline
            \textbf{Athanasius} & More Words \\ \hline
            \textbf{Babylon} & A large, ancient country that had control over the area, which is located in today's Iraq. This was the location that many were forced to migrate to during the exile.\\ \hline
            \textbf{Biblical Languages} & Hebrew, Aramaic, Greek.\\ \hline
            \textbf{Canon} & A measuring stick by which everything is judged. It is the authority in the religious community.\\ \hline
            \textbf{Constantine} & More Words \\ \hline
            \textbf{Council of Trent} & More Words \\ \hline
            \textbf{Dead Sea Scrolls} & Set of documents discovered in a cave in 1947 that contain the earliest known copies of the Hebrew Bible except Esther made just before the time of Jesus. The modern day copies of the Hebrew Bible match almost perfectly with the Dead Sea Scrolls' copies.\\ \hline
            \textbf{Deuterocanonical} & Having two sets of canon.\\ \hline
            \textbf{Exile} & More Words \\ \hline
            \textbf{Gnostics} & More Words \\ \hline
            \textbf{Hebrew Bible} & The Bible of the Jewish tradition. It includes the same 24 writings as the Old Testament except Samuel is separated into 1 and 2 Samuel and the Prophets are separated into the Twelve Minor Prophets.\\ \hline
            \textbf{Huldah} & More Words \\ \hline
            \textbf{Jerome} & More Words \\ \hline
            \textbf{Josephus} & More Words \\ \hline
            \textbf{Judah} & More Words \\ \hline
            \textbf{Marcion} & More Words \\ \hline
            \textbf{Messiah} & More Words \\ \hline
            \textbf{Muratorian Canon} & More Words \\ \hline
            \textbf{New Testament} & More Words \\ \hline
            \textbf{Old Testament} & The same as the Hebrew Bible. It has 24 books only because it combines 1 and 2 Samuel and the Twelve Minor Prophets into one.\\ \hline
            \textbf{Pentateuch} & The first five books of the Bible: Genesis, Exodus, Leviticus, Numbers, Deuteronomy.\\ \hline
            \textbf{Polytheism} & Believing in more than one god.\\ \hline
            \textbf{Second Temple Judaism} & The forms of Judaism that existed from 515 B.C.E. to 70 C.E. that begins after the temple in Jerusalem was built after being destroyed by the Babylonians in 587 B.C.E. and ends with the destruction in 70 C.E. by the Romans.\\ \hline
            \textbf{Septuagint} & The translation of the Hebrew Bible to Greek.\\ \hline
            \textbf{Torah} & The first five books of the Hebrew Bible: Genesis, Exodus, Leviticus, Numbers, Deuteronomy.\\ \hline
            \textbf{Vulgate} & The translation of the Hebrew and Greek Bibles to Latin.\\ \hline
        \end{longtable}
        \end{center}
    \chapter{From Then to Now}
    \chapter{Inspiration}

\part{What Is The Story Of The Hebrew Bible?}

    \chapter{The Pentateuch, Part I}
    \chapter{The Pentateuch, Part II}
    \chapter{The Israelites Tell Their Story}
    \chapter{``Thus Says The Lord''}
    \chapter{An Alternative Worldview}
    \chapter{Israel's Response to God}
    \chapter{Between the Testaments}

\part{What Is The Story Of The New Testament?}

    \chapter{The Gospels}
    \chapter{Four Views of One Jesus}
    \chapter{The Story Continues}
    \chapter{The Pauline Letters}
    \chapter{The Disputed Pauline Letters}
    \chapter{Hebrews and the General Epistles}
    \chapter{Revelation}

\part{Epilogue}

    \chapter{The Bible Today}

\end{document}
