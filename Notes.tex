\documentclass{article}
\usepackage{hyperref}
\hypersetup{colorlinks=false, linkcolor=black, urlcolor=black, linktoc=all}
\usepackage[british]{babel}
\usepackage[nodayofweek]{datetime}
\usepackage{helvet}
\renewcommand{\familydefault}{\sfdefault}

\title{Christian Story I}
\author{James Nhan}
\date{\longdate{\today}}

\begin{document}

\maketitle

\clearpage
\tableofcontents
\clearpage

\part{First Half}
\section{25 August, 2015}
    Overview of and orientation of the Bible. Focus on content, character, role in Christian faith, and the covenant themes of creation, fall, redemption, and new creation.
    \vspace{10pt}

    \noindent The following textbooks are required for the course:

    \begin{itemize}
        \item Sumney
        \item Anderson
        \item NOAB
    \end{itemize}

    \noindent Course Requirements:

    \begin{itemize}
        \item Exams: Two unit exams, one comprehensive final
        \item Quizzes: 6 (on readings)
        \item Biblical Message Assignment (oral presentation)
        \item Bibilical Story Paper
    \end{itemize}

    \noindent Homework:

    \begin{itemize}
        \item Read Sumney Ch. 1
    \end{itemize}

\section{27 August, 2015}
\centerline{History and Geography of the Bible}

    Israel and Mesopotamia are very important land bridges that connects \textbf{3} different continents between the deserts of the Middle East as well as the landmasses separated by the Mediterranian Sea.

    Under David and Solomon during the United Kingdom age from \textbf{1020} to \textbf{922}, the borders of Israel extended the furthest they ever had and ever have since. This is Israel's Golden Age where it was the most powerful and wealthy time of its history.

    In \textbf{722}, the country separated into the northern kingdom of Israel and the southern kingdom of Judah ruled by different groups. They did not always get along despite being related ethnically.

    Around \textbf{722}, Asyria wiped out Israel. The kingdom of Judah pays money to Asyria to avoid being destroyed.

    In succession, several large powers would take over each other. During the time of the Babylonian rule around \textbf{586} causing the exile and destruction of the Temple of Solomon. Those exiled were serving as slaves in Babylon. Around \textbf{536}, the Persian empire ruled. The difference of the Persian rule was that the exiles were allowed to return home. They were released to return to their life in Israel, but they were taxed heavily by the Persian empire. Around \textbf{333}, the Hellenistic period took
    over the Persian period until around \textbf{63} through the intertestamental stretch of time when Alexander the Great marched across the giant Persian empire until his death at age \textbf{33} due to sickness.
    The Maccabean family does resist the Hellenistic rule, but did not fight back.

    The Herodian family worked with the Roman family that would rule on behalf of the Romans in that region.

    The Second Temple Period refers to the period after the exile until \textbf{70 C.E.} when the temple was destroyed and replaced with the Dome of the Rock, a mosque.

    The Diaspora took place beginning at the start of the exile. Diaspora literally means something like ``\textbf{scattering}.''

\end{document}
