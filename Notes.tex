\documentclass{article}
\usepackage{hyperref}
\hypersetup{colorlinks=false, linkcolor=black, urlcolor=black, linktoc=all}
\usepackage[british]{babel}
\usepackage[nodayofweek]{datetime}
\usepackage{helvet}
\renewcommand{\familydefault}{\sfdefault}

\title{Christian Story I}
\author{James Nhan}
\date{\longdate{\today}}

\begin{document}

\maketitle

\clearpage
\tableofcontents
\clearpage

\part{First Half}
\section{25 August, 2015}
    Overview of and orientation of the Bible. Focus on content, character, role in Christian faith, and the covenant themes of creation, fall, redemption, and new creation.
    \vspace{10pt}

    \noindent The following textbooks are required for the course:

    \begin{itemize}
        \item Sumney
        \item Anderson
        \item NOAB
    \end{itemize}

    \noindent Course Requirements:

    \begin{itemize}
        \item Exams: Two unit exams, one comprehensive final
        \item Quizzes: 6 (on readings)
        \item Biblical Message Assignment (oral presentation)
        \item Biblical Story Paper
    \end{itemize}

    \noindent Homework:

    \begin{itemize}
        \item Read Sumney Ch. 1
    \end{itemize}

\section{27 August, 2015}
\centerline{History and Geography of the Bible}

    Israel and Mesopotamia are very important land bridges that connects \textbf{3} different continents between the deserts of the Middle East as well as the landmasses separated by the Mediterranian Sea.

    Under David and Solomon during the United Kingdom age from \textbf{1020} to \textbf{922}, the borders of Israel extended the furthest they ever had and ever have since. This is Israel's Golden Age where it was the most powerful and wealthy time of its history.

    In \textbf{722}, the country separated into the northern kingdom of Israel and the southern kingdom of Judah ruled by different groups. They did not always get along despite being related ethnically.

    Around \textbf{722}, Asyria wiped out Israel. The kingdom of Judah pays money to Asyria to avoid being destroyed.

    In succession, several large powers would take over each other. During the time of the Babylonian rule around \textbf{586} causing the exile and destruction of the Temple of Solomon. Those exiled were serving as slaves in Babylon. Around \textbf{536}, the Persian empire ruled. The difference of the Persian rule was that the exiles were allowed to return home. They were released to return to their life in Israel, but they were taxed heavily by the Persian empire. Around \textbf{333}, the Hellenistic period took
    over the Persian period until around \textbf{63} through the intertestamental stretch of time when Alexander the Great marched across the giant Persian empire until his death at age \textbf{33} due to sickness.
    The Maccabean family does resist the Hellenistic rule, but did not fight back.

    The Herodian family worked with the Roman family that would rule on behalf of the Romans in that region.

    The Second Temple Period refers to the period after the exile until \textbf{70 C.E.} when the temple was destroyed and replaced with the Dome of the Rock, a mosque.

    The Diaspora took place beginning at the start of the exile. Diaspora literally means something like ``\textbf{scattering}.''

\section{1 September, 2015}
\centerline{Hebrew Bible: Contents \texttt{\&} Organization}

    Aka Tanak = Torah (The Law), Neviim (Prophets), Kethuvim (Writings)

    The Law: 5 Books

    Prophets: 8 Books

    \begin{itemize}
        \item Former prophets (4 books)
        \item Latter prophets (4 books)
    \end{itemize}

    Writings: 11 books

    24 Total (OT is 39 total)

    (Old Testament) Apocrypha

\centerline{New Testament: Contents \texttt{\&} Organization}

    Gospels (4 books)

    Acts (1 book) \texttt{-} Actually several books titled the Acts of the Apostles like Peter.

    Pauline Epistles (13 books)

    Catholic Epistles (8 books)

    Apocalypse (1 book)

    27 total

\centerline{The Other ``Biblical History''}

    \begin{enumerate}
        \item Oral Tradition
        \item Written Documents
        \item Collection
        \item Canonization
        \item Translation
        \item Publication
        \item Textual Criticism
    \end{enumerate}

\centerline{Canonization}

    Hebrew Bible

    3 groupings = 3 stages

    Solidifed by ca. 100 CE

\section{3 September, 2015}

\centerline{\large \textbf{Quiz}}

\centerline{Tuesday, 8 September}

\noindent{Multiple Choice and Matching}

\noindent{Sumney's Key Terms: Ch 1, 2, 4}

\noindent{Biblical Books: Genesis to Deuteronomy}

        \indent{In order and spelled correctly}

\centerline{Canonization of the New Testament}

    The first Christians existed with no New Testament at all. They had proclamation, preaching, and a message. It takes time to have an authoratative canon.

    Eusebius of Caesarea \texttt{-} One of the original early church fathers in the early 4th century. Wrote ``Church History'' the history of the church in the first three centuries. First one.

    Athanasius of Alexandria \texttt{-} Bishop of Alexandria in Egypt. Lots of churches exist in Alexandria. In 367 C.E., he writes a letter to the churches with a list of books he considers authoratative, which is identical to what is in the modern New Testament Bible.

    Not until the Reformation, during the time of Martin Luther, do we have a major agreement on what books should be included in the New Testament.

\centerline{Translation and Publication}

    \begin{itemize}
        \item Targum: Hebrew to Aramaic (5th c. B.C.E.)
        \item Septuagint (LXX): Hebrew to Greek (3rd c. B.C.E.)
        \item Vulgate: Hebrew and Greek to Latin (5th c. C.E.)
    \end{itemize}

    Aramaic was the language used in exile by the Judahites. When they returned home, they brought Aramaic back with them. The new generation did not speak Hebrew, so they needed to translate everything in Hebrew to Aramaic.

    Septuagint means 70 (Latin). This was after the time of Alexander of Macedonia, so he brought the Greek language with him when he was conquering the world. This caused the Greek language to spread under the Hellenistic rulers. Seventy Hebrew scholars were called in in order to translate from Hebrew to Greek the Torah (probably still just the Torah at this time).

    Jerome was the translator. Vulgate means ``base'' or ``common''. It was the translation into the common language, which in Rome was Latin. It became the authoratative text for the Roman Catholic church.

\centerline{English Translation}

    In 1382, John Wycliffe makes the first English translation of the Bible based on the Vulgate.

    In 1525, William Tyndale made the first English translation of the New Testament from the Greek version, which became the basis for the King James Version.
    
    Formal Correspondence is a Word For Word translation from the source language to the destination language. It is nearly impossible to do this in translation, so most focus on this but also include Dynamic Equivalent translations.

    Dynamic Equivalent is a Thought For Thought translation from the source language to the destination language. This is trying to get the same idea, but without needing to match sentence structure and vocabulary.

\centerline{Textual Criticism}

    Textual Criticism is the study of ancient manuscripts.

    Notable manuscripts are the Dead Sea Scrolls discovered in the 1940's that date back to around the 1st c. B.C.E., which were extremely identical to the 10th c. C.E. manuscripts.

    Intentional Changes: The ending or endings of Mark.

    The original ending ended at ``\ldots they were afraid.''

    The shorter ending ended at ``\ldots they were afraid. And all that they ha been commanded them they told briefly to those around Peter. And afterward jesus himself sent out through them, from east to west, the sacred and imperishable proclamation of eternal salvation.''

    The longer ending ended even longer with several paragraphs more.

\centerline{The Torah}

    The Documentary Hypothesis says the Torah has four distinct sources: JEDP.\@

    \begin{enumerate}
        \item J = Yahwist
        \item E = Elohist
        \item D = Deuteronomist
        \item P = Priestly Source
    \end{enumerate}

    How do we distinguish between the sources?

    \begin{enumerate}
        \item Duplication or replication of material
        \item Variation in the ways of referring to God
        \item Contrasting perspectives of authors
        \item Variation in vocabulary and literary style
        \item Evidence of editorial activity
    \end{enumerate}

\centerline{Cosmic Origins}

    Genesis 1:

    \begin{itemize}
        \item Day 0: Water and Darkness
        \item Day 1: Light
        \item Day 2: Sky Dome
        \item Day 3: Earth, Sea, Plants
        \item Day 4: Sun, Moon
        \item Day 5: Fish, Birds
        \item Day 6a: Animals
        \item Day 6b: Man, Woman
        \item Day 7: (Sabbath)
    \end{itemize}

    Genesis 2:

    \begin{itemize}
        \item Earth, Heaven, Barren
        \item Man from Dust
        \item Garden, Plants, Trees
        \item Rivers
        \item Man in Garden
        \item Animals, Birds, Names
        \item Woman
        \item (Marriage)
    \end{itemize}

    Enuma Elish:

    \begin{itemize}
        \item Apsu, Tiamat (fresh water and salt water dieties, respectively)
        \item More Gods (children of Apsu and Tiamat)
        \item Marduk (also child)
        \item Creation of Monsters (Apsu and Tiamat's army to kill the other gods)
        \item Marduk becomes chief god if he can save other gods from monsters and parents
        \item Marduk defeats Apsu, Tiamat (last fight), and monsters
        \item Marduk creates Sky, Earth, and Sea
        \item Marduk creates Moon
        \item Marduk creates Humans as Slaves
    \end{itemize}

\section{10 September, 2015}

\centerline{The Fall: Genesis 3}

    What's the big idea of Genesis 3?

    \begin{enumerate}
        \item Sin: The cursing is different between the serpent and man.
        \item Power: God is threatened by an immortal, knowing man?
        \item Sex: Forbidden Fruit is sex.
    \end{enumerate}

    \centerline{The Flood: Genesis 6\texttt{-}9}

    Ancient Flood Narratives

    Noah

    \begin{itemize}
        \item God sends a flood
        \item Noah and household survive
        \item Build a boat shaped ark
        \item Animals on board
        \item God shuts the door
        \item 40 days and 40 nights it rains
        \item Rests on Mount Ararat
        \item Raven, dove (3x) sent out
        \item Sacrifice to God
        \item Rainbow
    \end{itemize}

    Gilgamesh (Ancient Mesopotamian polytheistic culture)

    \begin{itemize}
        \item Gods send flood (plural)
        \item Utnapishtim and household survive
        \item Cube shaped ark
        \item Animals on board
        \item Utnapishtim shuts the door
        \item 6 days and 6 nights it rains
        \item Rests on Mount Nisir
        \item Dove, swallow, raven sent out
        \item Sacrifice to gods
        \item Necklace
    \end{itemize}

    \centerline{Ancient Recreation Narratives}

    Noah

    \begin{itemize}
        \item The Deeps
        \item The Wind
        \item Waters reunite
        \item Classification of animals
        \item Humanity destroyed
        \item ``Be fruitful and multiply''
        \item Provision of food
    \end{itemize}

    Genesis 1

    \begin{itemize}
        \item The Deeps
        \item The Wind
        \item Waters recede
        \item Classification of animals
        \item Humanity created
        \item ``Be fruitful and multiply''
        \item Provision of food
    \end{itemize}

    \centerline{Genesis 1\texttt{-}11}

    \begin{itemize}
        \item Catalyst
        \item Rebellion
        \item Confrontation
        \item Rationalization
        \item Alienation
        \item Restoration
    \end{itemize}

    What is a ``covenant''?

    Put simply: An agreement between two parties; a contract.

    Put biblically: The relationship between God and God's people, characterized by mutual love and responsibility.

\section{15 September, 2015}

    \centerline{Covenantal Relationship}

    \begin{itemize}
        \item Creation (J \& P)
        \item Image of God (P)
        \item First Sacrifice (J)
        \item Cain's Mark (J)
        \item Seth (J)
        \item Genealogies (J \& P)
        \item Noah's Ark (J \& P)
        \item Promise of Life (J)
        \item Rainbow (P)
        \item Blessing on Shem (P)
    \end{itemize}

    \centerline{Tower of Babel}

    Ziggurat at Ur

    \centerline{Genesis 12\texttt{-}50: The Highlights}

    Elements of the Covenant

    \begin{itemize}
        \item A great nation from Abraham's descendants
        \item A blessing to all nations through Abraham
        \item A land to call their own
    \end{itemize}

    The signs of the covenant

    \begin{itemize}
        \item The meeting with Melchizedek
        \item The covenant ceremony
        \item Circumcision
    \end{itemize}

    The Covenant and Jacob

    \noindent Birth of Jacob \\
    \indent Jacob gets Esau's birthright \\ 
    \indent \indent Conflict: Isaac's household and Abimelech \\
    \indent \indent \indent Jacob and Esau split \\
    \indent \indent \indent \indent Jacob and Bethel (house of God) \\
    \indent \indent \indent \indent \indent Jacob stays with Laban \\
    \indent \indent \indent \indent Jacob at Penuel (face of God) \\
    \indent \indent \indent Jacob and Esau reconciled \\
    \indent \indent Conflict: Jacob's household and Shechem \\
    \indent Jacob returns to his birthright \\
    Death of Isaac

    \begin{itemize}
        \item Prophet = Proclaims God's word \& will
        \item Priest = Intercedes for God's people
        \item King = Leads people in God's path
    \end{itemize}

    Pharaoh said to his servants, ``Can we find anyone else like this, one in whom is the spirit of God?'' So Pharaoh said to Joseph, ``Since God has shown you all this, there is no one so discerning and wise as you. You shall be over my house, and all my people shall order themselves as you command.''

\section{17 September, 2015}

    Quiz on Tuesday over Sumney Key Terms chapter 5 and 6, books Joshua through Esther on Sumney Page 11.

    Moses' life parallels with Sargon of Agade. Both were placed in a basket of reeds sealed with itumen, floated down a river, found and raised by another, and became a great leader.

    Moses starts in Egypt and kills a slavemaster. Is exiled from Egypt and he goes to Midian. Moses meets a woman named Zipporah, who he defends from some men at a well. He marries and starts a family with him. Her father is a priest in Midian. Sees the burning bush (a theophany) on a mountain while herding sheep one day. During this vision, God reveals his divine name: ehyeh asher ehyeh, or yahweh. He returns to Egypt and asks the Pharaoh to release his people. He is asked who sent
    him with this message. He replies ``the God who exists.'' This is one of the first instances of monotheism. Most everyone was a polytheist.

    We do not know which Pharaoh it was. It depends on when the Exodus is dated, but it is heavily debated.

    \begin{enumerate}
        \item Nile to Blood targets Khnum, the creator of water and life, Hapi, the Nile god, and Osiris, the Nile as his bloodstream.
        \item Frogs target Heket, the goddess of childbirth whose symbol was the frog.
        \item Gnats
        \item Flies
        \item Cattle disease targets Hathor, the mother and sky goddess whose symbol was the cow, and Apis, the bull god.
        \item Boils
        \item Hail targets Seth, the god of wind and storm.
        \item Locusts targets Isis, the goddess of life, and Min, the goddess of fertility and vegetation, protector of crops.
        \item Darkness Amon\texttt{-}Ra, Atum, Horus, the sun dieties.
        \item Death of the firstborn targets Osiris, the judge of the dead and patron diety of the Pharaoh.
    \end{enumerate}

    Finally, we have the passover where the blood of a lamb is smeared over the doorpost of the Israelites.

    The first major stop after the exodus is at Mt. Sinai. Moses receives the Decalogue, or the Ten Commandments, the Covenant Code, the Holiness Code, the Deuteronomic Code, and the Shema.

    The Golden Calf

    Spying out the Land

    The Waters of Meribah

\section{22 September, 2015}

    \centerline{The Former Prophets: Part I}

    From the occupation to monarchy.

    Deuteronomistic History:

    \begin{itemize}
        \item Deuteronomy and Former Prophets (Joshua, Judges, Samuel, Kings)
        \item Complete history from Exodus to Exile
        \item{Deuteronomistic Themes \\ 
            \begin{itemize}
                \item Disobedience and Judgement (Martin Noth)
                \item Covenant and Grace (Gerhard von Rad)
                \item Repentance and Restoration (H. W. Wolff)
                \item Prophet, Priest, King
            \end{itemize}}
    \end{itemize}

    \centerline{Joshua}

    \begin{itemize}
        \item Military Campaigns
        \item Distribution of the Land
        \item Joshua's Farewells
    \end{itemize}

    Gilgal: A new beginning

    Build a monument with 12 riverbed stones.

    Gibeath\texttt{-}haaraloth

    Passover resumes.

    Manna stops.

    Mt. Ebal (Shechem): The covenant
    
    Build another monumnet.

    Faithful to the covenant.

    \centerline{Judges}

    \begin{itemize}
        \item The Problem
        \item The Solution
        \item The Migration of Dan
        \item The Attack on Benjamin
    \end{itemize}

\end{document}
