\documentclass{article}
\usepackage{hyperref}
\hypersetup{colorlinks=false, linkcolor=black, urlcolor=black, linktoc=all}
\usepackage[british]{babel}
\usepackage[nodayofweek]{datetime}
\usepackage{helvet}
\renewcommand{\familydefault}{\sfdefault}

\title{Christian Story I}
\author{James Nhan}
\date{\longdate{\today}}

\begin{document}

\maketitle

\clearpage
\tableofcontents
\clearpage

\part{First Half}
\section{25 August, 2015}
    Overview of and orientation of the Bible. Focus on content, character, role in Christian faith, and the covenant themes of creation, fall, redemption, and new creation.
    \vspace{10pt}

    \noindent The following textbooks are required for the course:

    \begin{itemize}
        \item Sumney
        \item Anderson
        \item NOAB
    \end{itemize}

    \noindent Course Requirements:

    \begin{itemize}
        \item Exams: Two unit exams, one comprehensive final
        \item Quizzes: 6 (on readings)
        \item Biblical Message Assignment (oral presentation)
        \item Biblical Story Paper
    \end{itemize}

    \noindent Homework:

    \begin{itemize}
        \item Read Sumney Ch. 1
    \end{itemize}

\section{27 August, 2015}
\centerline{History and Geography of the Bible}

    Israel and Mesopotamia are very important land bridges that connects \textbf{3} different continents between the deserts of the Middle East as well as the landmasses separated by the Mediterranian Sea.

    Under David and Solomon during the United Kingdom age from \textbf{1020} to \textbf{922}, the borders of Israel extended the furthest they ever had and ever have since. This is Israel's Golden Age where it was the most powerful and wealthy time of its history.

    In \textbf{722}, the country separated into the northern kingdom of Israel and the southern kingdom of Judah ruled by different groups. They did not always get along despite being related ethnically.

    Around \textbf{722}, Asyria wiped out Israel. The kingdom of Judah pays money to Asyria to avoid being destroyed.

    In succession, several large powers would take over each other. During the time of the Babylonian rule around \textbf{586} causing the exile and destruction of the Temple of Solomon. Those exiled were serving as slaves in Babylon. Around \textbf{536}, the Persian empire ruled. The difference of the Persian rule was that the exiles were allowed to return home. They were released to return to their life in Israel, but they were taxed heavily by the Persian empire. Around \textbf{333}, the Hellenistic period took
    over the Persian period until around \textbf{63} through the intertestamental stretch of time when Alexander the Great marched across the giant Persian empire until his death at age \textbf{33} due to sickness.
    The Maccabean family does resist the Hellenistic rule, but did not fight back.

    The Herodian family worked with the Roman family that would rule on behalf of the Romans in that region.

    The Second Temple Period refers to the period after the exile until \textbf{70 C.E.} when the temple was destroyed and replaced with the Dome of the Rock, a mosque.

    The Diaspora took place beginning at the start of the exile. Diaspora literally means something like ``\textbf{scattering}.''

\section{1 September, 2015}
\centerline{Hebrew Bible: Contents \texttt{\&} Organization}

    Aka Tanak = Torah (The Law), Neviim (Prophets), Kethuvim (Writings)

    The Law: 5 Books

    Prophets: 8 Books

    \begin{itemize}
        \item Former prophets (4 books)
        \item Latter prophets (4 books)
    \end{itemize}

    Writings: 11 books

    24 Total (OT is 39 total)

    (Old Testament) Apocrypha

\centerline{New Testament: Contents \texttt{\&} Organization}

    Gospels (4 books)

    Acts (1 book) \texttt{-} Actually several books titled the Acts of the Apostles like Peter.

    Pauline Epistles (13 books)

    Catholic Epistles (8 books)

    Apocalypse (1 book)

    27 total

\centerline{The Other ``Biblical History''}

    \begin{enumerate}
        \item Oral Tradition
        \item Written Documents
        \item Collection
        \item Canonization
        \item Translation
        \item Publication
        \item Textual Criticism
    \end{enumerate}

\centerline{Canonization}

    Hebrew Bible

    3 groupings = 3 stages

    Solidifed by ca. 100 CE

\section{3 September, 2015}

\centerline{\large \textbf{Quiz}}

\centerline{Tuesday, 8 September}

\noindent{Multiple Choice and Matching}

\noindent{Sumney's Key Terms: Ch 1, 2, 4}

\noindent{Biblical Books: Genesis to Deuteronomy}

        \indent{In order and spelled correctly}

\centerline{Canonization of the New Testament}

    The first Christians existed with no New Testament at all. They had proclamation, preaching, and a message. It takes time to have an authoratative canon.

    Eusebius of Caesarea \texttt{-} One of the original early church fathers in the early 4th century. Wrote ``Church History'' the history of the church in the first three centuries. First one.

    Athanasius of Alexandria \texttt{-} Bishop of Alexandria in Egypt. Lots of churches exist in Alexandria. In 367 C.E., he writes a letter to the churches with a list of books he considers authoratative, which is identical to what is in the modern New Testament Bible.

    Not until the Reformation, during the time of Martin Luther, do we have a major agreement on what books should be included in the New Testament.

\centerline{Translation and Publication}

    \begin{itemize}
        \item Targum: Hebrew to Aramaic (5th c. B.C.E.)
        \item Septuagint (LXX): Hebrew to Greek (3rd c. B.C.E.)
        \item Vulgate: Hebrew and Greek to Latin (5th c. C.E.)
    \end{itemize}

    Aramaic was the language used in exile by the Judahites. When they returned home, they brought Aramaic back with them. The new generation did not speak Hebrew, so they needed to translate everything in Hebrew to Aramaic.

    Septuagint means 70 (Latin). This was after the time of Alexander of Macedonia, so he brought the Greek language with him when he was conquering the world. This caused the Greek language to spread under the Hellenistic rulers. Seventy Hebrew scholars were called in in order to translate from Hebrew to Greek the Torah (probably still just the Torah at this time).

    Jerome was the translator. Vulgate means ``base'' or ``common''. It was the translation into the common language, which in Rome was Latin. It became the authoratative text for the Roman Catholic church.

\centerline{English Translation}

    In 1382, John Wycliffe makes the first English translation of the Bible based on the Vulgate.

    In 1525, William Tyndale made the first English translation of the New Testament from the Greek version, which became the basis for the King James Version.
    
    Formal Correspondence is a Word For Word translation from the source language to the destination language. It is nearly impossible to do this in translation, so most focus on this but also include Dynamic Equivalent translations.

    Dynamic Equivalent is a Thought For Thought translation from the source language to the destination language. This is trying to get the same idea, but without needing to match sentence structure and vocabulary.

\centerline{Textual Criticism}

    Textual Criticism is the study of ancient manuscripts.

    Notable manuscripts are the Dead Sea Scrolls discovered in the 1940's that date back to around the 1st c. B.C.E., which were extremely identical to the 10th c. C.E. manuscripts.

    Intentional Changes: The ending or endings of Mark.

    The original ending ended at ``\ldots they were afraid.''

    The shorter ending ended at ``\ldots they were afraid. And all that they ha been commanded them they told briefly to those around Peter. And afterward jesus himself sent out through them, from east to west, the sacred and imperishable proclamation of eternal salvation.''

    The longer ending ended even longer with several paragraphs more.

\end{document}
